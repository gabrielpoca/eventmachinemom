%%%%%%%%%%%%%%%%%%%%%%%%%%%%%%%%%%%%%%%%%
% Simple Sectioned Essay Template
% LaTeX Template
%
% This template has been downloaded from:
% http://www.latextemplates.com
%
% Note:
% The \lipsum[#] commands throughout this template generate dummy text
% to fill the template out. These commands should all be removed when 
% writing essay content.
%
%%%%%%%%%%%%%%%%%%%%%%%%%%%%%%%%%%%%%%%%%

%----------------------------------------------------------------------------------------
%	PACKAGES AND OTHER DOCUMENT CONFIGURATIONS
%----------------------------------------------------------------------------------------

\documentclass[12pt]{article} % Default font size is 12pt, it can be changed here

\usepackage{natbib}
\bibliographystyle{plainnat}
\usepackage{url}


\usepackage[portuges]{babel}
\usepackage[utf8]{inputenc} % Acentuação
\usepackage{indentfirst}
%\setlength{\parskip}{\baselineskip} % Space between paragraphs
\usepackage{parskip}% http://ctan.org/pkg/parskip

\usepackage{geometry} % Required to change the page size to A4
\geometry{a4paper} % Set the page size to be A4 as opposed to the default US Letter
% For right aligend descriptions
\usepackage{calc}  
\usepackage{enumitem} 

\usepackage{graphicx} % Required for including pictures

\usepackage{float} % Allows putting an [H] in \begin{figure} to specify the exact location of the figure
\usepackage{wrapfig} % Allows in-line images such as the example fish picture

\usepackage{lipsum} % Used for inserting dummy 'Lorem ipsum' text into the template

\linespread{1.2} % Line spacing

%\setlength\parindent{0pt} % Uncomment to remove all indentation from paragraphs

\graphicspath{{./pictures/}} % Specifies the directory where pictures are stored

\begin{document}

%----------------------------------------------------------------------------------------
%	TITLE PAGE
%----------------------------------------------------------------------------------------

\begin{titlepage}

  \newcommand{\HRule}{\rule{\linewidth}{0.5mm}} % Defines a new command for the horizontal lines, change thickness here

  \center % Center everything on the page

  \includegraphics{logo}\\[1cm] % Include a department/university logo - this will require the graphicx package

  \textsc{\LARGE Universidade do Minho}\\[1.5cm] % Name of your university/college
  \textsc{\Large Sistemas Distribuídos}\\[0.5cm] % Major heading such as course name
  %\textsc{\large Minor Heading}\\[0.5cm] % Minor heading such as course title

  \HRule \\[0.4cm]
  { \huge \bfseries Projecto Integrado}\\[0.1cm] % Title of your document
  { \small Um sistema de troca de mensagens com Ruby e Websockets }
  \HRule \\[0.8cm]

  \begin{flushleft}
    Gabriel Poça, PG22804 \\
    Rafael Remondes, PG22801\\
  \end{flushleft}


  {\large \today}\\[3cm] % Date, change the \today to a set date if you want to be precise

  \begin{abstract}
    Procura explorar as potencialidades da linguagem Ruby de modo a oferecer
    a clientes web um plataforma de troca de mensagens.
  \end{abstract}


  \vfill % Fill the rest of the page with whitespace

\end{titlepage}

%----------------------------------------------------------------------------------------
%	TABLE OF CONTENTS
%----------------------------------------------------------------------------------------

\tableofcontents % Include a table of contents

\newpage % Begins the essay on a new page instead of on the same page as the table of contents 

%----------------------------------------------------------------------------------------
%	INTRODUCTION
%----------------------------------------------------------------------------------------
\section{Introdução}

Este documento serve o propósito de documentar o estudo e desenvolvimento realizado no projeto integrado da unidade curricular de sistemas distribuídos.

O que se pretende com este projeto é desenvolver um \textit{message-oriented middleware} (MOM) distribuído que suporte comunicação por websockets. 
A motivação para o projeto está tanto na crescente utilização de clientes web como na ausência de suporte nativo para websockets por parte dos MOM mais populares. Soluções como o RabbitMQ e ZeroMQ fornecem plugins que criam uma ponte entre websockets e o protocolo Simple Text Oriented Message Protocol.

Uma vez que se pretende explorar uma área extensa algumas funcionalidades presentes em grande parte dos MOM têm de ser deixadas de parte, focando a implementação nas caracteristicas que suportam a análise que se pretende neste projeto:

\begin{itemize}
\item Suportar múltiplos clientes.
\item Permitir que os clientes subscrevam canais.
\item Permitir criar canais persistentes.
\end{itemize}

No contexto deste projeto o termo \textbf{canal} canal significa \textbf{message queue}, este assunto será explorado com mais detalhe na secção~\ref{sec:mom}.


As próximas secções desta introdução introduzem conceitos e tecnologias utilizadas com o objetivo de suportar as análises posteriores no relatório.



\subsection{Websockets}
WebSockets é um protocolo que permite umq comunicação \textit{full-duplex} sobre um único \textit{socket} TCP. Foi desenhado para ser usado em \textit{browsers} e em servidores que suportem o HTML 5.\\
Historicamente, ao criar uma aplicação \textit{web} em que fosse necessário uma comunicação bidireccional entre cliente e servidor, isso implicava um abuso do HTTP para o \textit{poll} do servidor para actualizações enquanto enviava notificações para chamadas HTTP distintas. Esta abordagem levantava vários problemas, entre os quais o facto de o servidor ser forçado a usar um número diferente de conexões TCP subjacentes para cada cliente assim como facto de o protocolo ter um grande \textit{overhead}.\\
A solução mais simples para o problema seria usar apenas uma conxeão TCP para o tráfego em ambas as direcções. Essa solução é apresentada pela API fornecida pelos \textit{WebSockets} ao providenciar uma alternativa ao usual HTTP \textit{polling} para comunicação nas duas direcções.\\
Esta técnica de \textit{WebSockets} pode ser usada para várias aplicações Web, especialmente as que necessitam de constante trocas de pacotes em tempo real como jogos on-line. O protocolo websocket é um tecnologia incluída no html5 e é suportada actualmente pela maior parte dos browsers. O seu esquema de de url é \textit{ws} ou \textit{wss} para comunicações seguras. \\

A conceção inicial da web considerou apenas a comunicação cliente-servidor num sentido apenas. Actualmente o HTML5 procura corrigir esta entrave, contudo ainda muitos projectos utilizam \textit{long-polling} para simular a comunicação cliente-servidor.\\

Actualmente os browsers são actualizados regularmente e suportam a API de comunicação do HTML5.
\subsubsection{\textit{Long pulling}}

\begin{figure}[H]
\centering
\includegraphics[width=0.9\textwidth]{longpolling-architecture.png}
\caption{Esquema de \textit{long pulling}}
\label{fig:long_pulling}
\end{figure}

Um cliente (browser) envia por HTTP um pedido para o servidor com o identificador do utilizador (por exemplo) e do estado actual. No servidor é criado um processo que repetidamente verifica na base de dados se existe um estado novo. Quando existe um novo estado o cliente recebe e envia um novo pedido ao servidor.

\subsubsection{\textit{Server-Sent Events}}

\begin{figure}[H]
\centering
\includegraphics[width=0.9\textwidth]{sse-architecture.png}
\caption{Esquema de \textit{server-sent events}}
\label{fig:sse-architecture}
\end{figure}

Um cliente (browser) faz um pedido ao servidor. O servidor responde com o último estado na base de dados. O cliente recebe a resposta e em três segundos (por exemplo) envia um novo pedido.

\subsubsection{Websockets}

\begin{figure}[H]
\centering
\includegraphics[width=0.9\textwidth]{websocket-architecture.png}
\caption{Esquema de \textit{websockets}}
\label{fig:websockets-architecture}
\end{figure}

Um cliente notifica o servidor de websockets de um evento. O servidor imediamente notifica todos os clientes ativos do evento. Este processo pode envolver filtros e subscrição de eventos.







\subsection{Message Oriented Middleware - MoM}

\label{sec:mom}
Entende-se por \textit{message oriented middleware} (MOM) como um método de comunicação entre componentes de sistemas distribuídos.\\

Um \textit{Message Oriented Middleware}, ou simplesmente MoM é uma arquitectura que fornece uma camada entre as aplicaçães, substituindo a comunicação directa entre as mesmas por um sistema de troca de mensagens.\\
Uma implementação \textit{MoM} oferece uma API que fornece um nível de abstracção capaz de aumentar a portabilidade, interoperabilidade e flexibilidades das aplicações que correm sobre a MOM.Os programadores são assim libertados dos detalhes das várias plataformas e protocolos, reduzindo a complexidade das comunicações das suas aplicações.\\
A comunicação é efectuada através de troca de mensagens pela estrutura típica cliente/servidor(usando \textit{broadcast} ou \textit{multicast}) ou entre pilhas mantidas por gestores locais. Os gestores de pilhas são poderosos em termos de aplicabilidade e versatilidade.\\ 
Os sistemas que usam MOM providenciam comunicação distribuída com base num modelo de interacção assíncrono. Os participantes do sistema não precisam de bloquear e esperar numa mensagem enviada, eles podem continuar a processar. Isso permite a entrega de mensagens quando o receptor ou emissor não estão activos ou disponíveis na altura da execução. Uma aplicação que envia mensagens não tem a garantia que a sua mensagem vai ser lida, nem  a garantia de quanto tempo vai demorar até que seja entregue. 
subsection{Padrões de troca de mensagens }
\begin{itemize}
\item \textbf{Ponto a Ponto}\\
O modelo de mensagens ponto a ponto fornece uma troca assíncrona de mensagens entre aplicações. Neste modelo, as mensagens de um cliente produtor são encaminhadas para um cliente consumidor através de uma \textit{queue}. O mecanismo mais comum de \textit{queue} é uma \textit{queue} FIFO, na qual as mensagens são ordenadas conforme a ordem em que são recebidas pelo sistema de mensagens, assim que são consumidas são removidas do topo da \textit{queue}. 
Apesar de não existir uma restrição para o número de clientes que podem publicar numa \textit{queue}, existe normalmente apenas um cliente consumidor, apesar de não ser um requisito muito rígido. Cada mensagem que é entregue apenas uma vez a apenas um receptor. O modelo permite que múltiplos receptores possam se conectar à queue, mas apenas um dos receptores vai consumir a mensagem. No modelo ponto a ponto, as mensagens são sempre entregues e são guardadas na \textit{queue} até que um consumidor estar preparado para a consumir.  
\item \textbf{Publish/Subscribe}
O mecanismo  de \textit{Publish/Subscribe} é um mecanismo muito poderoso, usado para desiminar informação entre produtores e consumidores anonímos de mensagens. Podem ser mecanismos de distribuição de um para um ou de muitos para muitos, permitem a um simples consumidor enviar e receber mensagens de, potencialmente, centenas de milhares de utilizadores.\\
No modelo \textit{publish/subscribe}, a aplicação emissora e receptora é livre da necessidade de perceber alguma coisa sobre a aplicação alvo.  Só tem necessidade de enviar a informação para um destino dentro da máquina \textit{publish/subscribe}. A máquina vai enviar prosteriormente para o consumidor. Os cliente produzem mensagens para um tópico especifico ou para um canal que são subscritos por cliente que pretendem consumir essas mensagens. O serviço mapeia as mensagens para os consumidores consoante os tópicos que eles estão interessados. Dentro do modelo \textit{publish/subscribe}, não existe a restrição no papel de um cliente, ele pode ser tanto consumidor como produtor de um tópico/canal.\\
\end{itemize}

\section{Ferramentas}

As próximas secções introduzem ferramentas utilizadas no projecto, desde bibliotecas ruby e javascript a base de dados.

\subsection{Ruby}

Ruby é uma linguagem de programação intrepertada com suporte para diferentes paradigmas (funcional, orientado a objectos e imperativo). Desde o lançamento em 1995 Ruby tem crescido em comunidade e potencial. O seu criador, Yukihiro Matsumoto, pretendia uma linguagem que qualquer programador pudesse aperciar:

``Ruby is simple in appearance, but is very complex inside, just like our human body.'' \cite{matz}

Ruby encontra-se actualmente na versão 2.0.0-p195 no entanto o projecto foi desenvolvido na versão \textbf{2.0.0p0}.

\subsubsection{\textit{Global Interpreter Lock}}

Existem diversos intrepertadores para Ruby sendo os mais importantes \textbf{Matz's Ruby Interpreter (MRI)}, em homenagem ao criador, e \textbf{JRuby}, implementado no topo da Java Virtual Machine. 
A diferença mais relevante entre ambos para este projecto é o \textit{Global Interpreter Lock} (GIL) que existe no MRI. O GIL é uma camada responsável por proteger o intrepertador contra código \textit{non thread-safe}.

A figura~\ref{fig:ruby-gil} apresenta uma comparação entre três versões do Ruby. Na versão 1.8 o intrepertador Ruby possuí apenas uma thread do sistema para execussão. Já na versão 1.9, e também na versão 2 ainda que não seja visível na imagem, várias threads do sistema são alocadas ao intrepertador, o que parece prometer paralelismo de execussão. No entanto em ambos os casos existe a camada do GIL que protegendo contra a execussão de código \textit{non thread-safe} permite que apenas uma thread seja executada de cada vez pelo processador, ou seja, ambas as versões do Ruby correm num core do CPU apenas.
Por outro lado a implementação em JRuby não contém a camada GIL o que abre as portas ao paralelismo das aplicações em Ruby.

\begin{figure}[H]
\centering
\includegraphics[width=0.9\textwidth]{xruby_gil.png}
\caption{\textit{Global Interpreter Lock}}
\label{fig:ruby-gil}
\end{figure}

Parecendo um cenário desvantajoso para o MRI é de notar que existem soluções para contornar este problema. Se se pensar em processos em vez de threads por procurar-se outros meios de repartir trabalho. Decompondo a aplicação e adicionando meios de comunicação entre processos (Starling, RabbitMQ, outros) consegue-se que multiplos processos da mesma aplicação executem concurrentemente.

\subsubsection{RubyGems}

A aplicação do servidor, em Ruby, encontra-se no formato RubyGem, geralmente denominado por gem. O software RubyGem permite que facilmente se descarregue, instale e manipule gems num sistema.\cite{rubygems} O que se pretende com esta preocupação é permitir que a aplicação possa facilmente ser distribuída e desse modo contribuir para a comunidade Ruby.

Em modo muito básico uma gem é composta pelo seguinte:

\begin{description}
\item[Gemspec] Ficheiro que identifica as dependência da gem. No caso deste projecto o nome do ficheiro é \textit{eventmachinemom.gemspec}.
\item[Código da aplicação] Código a executar pela aplicação, encontra-se na pasta \textit{lib}.
\item[Testes] Testes da aplicação. No caso deste projecto os testes são \textbf{RSpec} e encontram-se na pasta \textit{spec}.
\item[Excutáveis] Ficheiros executáveis que são instalados no sistema que podem ser invocados pelo utilizador ou outro software. Encontram-se na pasta \textit{bin}.
\end{description}

\subsubsection{EventMachine}

EventMachine é uma biblioteca para Ruby que implementa \textit{event-drive I/O}. Ruby não foi concebido para executar concorrentemente nem para suportar programação por eventos.




A biblioteca é utilizada em servidores de eventos; servir clientes assincronos por diferentes protocolos; proxys. \textbf{EventMachine Websockets}

\subsubsection{ActiveRecord}
Esta ferramenta é já parte do cliente.
Active Record é um biblioteca Ruby que establece uma ligação entre classes e tabelas de bases de dados relacionais sem grande configuração. Esta biblioteca uma classe base que quando é extendida establece um mapeamento entre a nova classe e um tabela existente na base de dados. A biblioteca foi desenvolvida inicialmente para a framework Ruby on Rails sendo o base do que se chama models.

É importante referir que o uso desta biblioteca permite que não se escreva SQL de de operações sobre a base de dados à excepção da criação de tabelas sendo esta processo gerado por operações em Ruby.

\subsection{PostgreSQL}

No primeiro esquema conceptual da arquitectura da aplicação planeou-se utilizar SQLite, cada broker teria a sua base de dados e depois haveria sincronização entre todos. 

A primeira abordagem além de complexa não explora o \textbf{controlo de concurrência} que grande parte das base de dados oferece como garantido. Como tal a nova abordagem passa por centralizar a base de dados e todos os servidores trabalham sobre a mesma conseguindo-se sincronização e controlo de concurrência.

\subsection{JQuery}

jQuery é uma biblioteca JavaScript que facilita a manipulação de HTML, eventos, animação, Ajax com uma API transversal aos browsers. Juntamente com a biblioteca JQuery faz-se uso da biblioteca \textbf{The Graceful WebSocket} que permite fornece um sistema de fallback para websockets.



\section{Servidor}

Nesta secção não se pretende analisar o servidor mas os broker e as respectivas componentes, descrevendo as respectivas responsabilidades e comportamento.
Relembra-se que o servidor é constituído por um numero aleatório de brokers e uma base de dados.

\subsection{Broker}

Os brokers são instâncias de uma aplicação desenvolvida em Ruby. Cada instância difere apenas nos endereços de comunicação: um para comunicação com clientes e outro para comunicação com os restantes brokers.
Sendo que cada broker executa num processo Ruby justifica-se que na mesma máquina se executem diversos brokers de forma a tirar partido de paralelismo de processamento (ver secção~\ref{sec:ruby}).
Note-se que cada broker executa numa thread apenas uma vez que a aplicação executa no no paradigma orientado a eventos fazendo uso da ferramenta EventMachine (ver secção~\ref{sec:eventmachine}).

Toda a comunicação do servidor funciona sobre websockets. Tanto com clientes como entre os brokers (à exceção evidente da comunicação com a base de dados). A decisão de utilizar websockets para comunicação entre os brokers é questionável, outros protocolos como o Advanced Message Queuing Protocol\footnote{http://en.wikipedia.org/wiki/Advanced\_Message\_Queuing\_Protocol} são uma solução eventualmente mais estável uma vez que é adotado por imensas ferramentas relacionadas.
Contudo o único requisito quanto à comunicação entre brokers neste projeto passa por garantir comunicação não bloqueante, como tal websockets é uma tecnologia que se utiliza na comunicação com clientes e não existe impedimento ao uso na comunicação entre brokers Motivos que revelem outros protocolos como mais adequados podem surgir em situações que se entenda que é necessário diminuir tempos de comunicação ou algo semelhante.

\subsubsection{Componentes}
Como se entende da introdução a esta secção cada broker garante dois serviços de websockets:

\begin{itemize}
\item um para comunicação com os clientes. Será identificado ao longo do relatório como \textbf{serviço de comunicação}).
\item um para comunicação com os restantes brokers. Será identificado ao longo do relatório como \textbf{serviço de sincronização}).
\end{itemize}

Ainda que execute dois serviços de websockets que atendem em portas diferentes os mesmos são servidor pela mesma thread num ambiente orientado a eventos. Esta decisão advém do facto de que utilizado o MRI não existe concorrência real entre threads (a justificação deste facto está na secção~\ref{sec:gil}).

O serviço de comunicação com os clientes responde aos seguintes eventos de um cliente:

\begin{enumerate}
\item \textbf{nova ligação}. O serviço é recolhe o próximo identificador único da base de dados e envia ao cliente. Este identificador serve apenas para o cliente poder assinar as suas mensagens ou reconhecer as mensagens de outro cliente (este identificador não tem utilidade ao servidor).
\item \textbf{subscrição de canal}. O serviço procura o canal ou cria um novo e devolve ao cliente uma mensagem de sucesso.
\item \textbf{remoção da subscrição de um canal}. O serviço remove a subscrição do cliente.
\item \textbf{mensagem para diversos canais}. O serviço difunde a mensagem pelos respetivos canais e submete para o serviço de sincronização.
\end{enumerate}

\hl{}

\textbf{Novo servidor}
\begin{enumerate}
\item Iniciar o servidor que recebe ligações dos restantes brokers.
\item Registar o endereço do servidor na base de dado
\item Criar a ligação a todos os brokers que estão ativos na base de dados.
\item Utilizar cada ligação para informar da atualização da lista de brokers.
\item Iniciar o servidor de websockets.
\end{enumerate}



%----------------------------------------------------------------------------------------
%	CONCLUSION
%----------------------------------------------------------------------------------------

\section{Conclusion} 

%----------------------------------------------------------------------------------------
%	BIBLIOGRAPHY
%----------------------------------------------------------------------------------------

%\bibliographystyle{plain}
\begingroup
\raggedright
\bibliography{bibliography.bib}
\endgroup

%----------------------------------------------------------------------------------------

\end{document}
