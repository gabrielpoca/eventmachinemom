\section{Introdução Teórica}

Esta secção serve de introdução teórica aos conceitos utilizados ao longo do relatório. O objectivo é relembrar o fundamental e establecer a terminologia a utilizar no desenvolvimento do relatório.

% ---------------------- Message Oriented Middleware

\subsection{Message oriented middleware (MOM)}
\label{sec:mom}
Entende-se por \textit{message oriented middleware} (MOM) como um método de comunicação entre componentes de sistemas distribuídos.

\begin{description}
  \item[Assincrono] \hfill \\
  	Permite que um cliente não bloqueie enquanto espera por resposta.
  \item[Second]
  \item[Third] The third etc \ldots
\end{description}

No contexto deste relatório entede-se por MOM como uma camada de comunicação que permite a várias aplicações comunicar ignorando as especificidades de cada uma.

\subsection{Message Oriented Middleware - MoM}

\label{sec:mom}
Entende-se por \textit{message oriented middleware} (MOM) como um método de comunicação entre componentes de sistemas distribuídos.\\

Um \textit{Message Oriented Middleware}, ou simplesmente MoM é uma arquitectura que fornece uma camada entre as aplicaçães, substituindo a comunicação directa entre as mesmas por um sistema de troca de mensagens.\\
Uma implementação \textit{MoM} oferece uma API que fornece um nível de abstracção capaz de aumentar a portabilidade, interoperabilidade e flexibilidades das aplicações que correm sobre a MOM.Os programadores são assim libertados dos detalhes das várias plataformas e protocolos, reduzindo a complexidade das comunicações das suas aplicações.\\
A comunicação é efectuada através de troca de mensagens pela estrutura típica cliente/servidor(usando \textit{broadcast} ou \textit{multicast}) ou entre pilhas mantidas por gestores locais. Os gestores de pilhas são poderosos em termos de aplicabilidade e versatilidade.\\ 
Os sistemas que usam MOM providenciam comunicação distribuída com base num modelo de interacção assíncrono. Os participantes do sistema não precisam de bloquear e esperar numa mensagem enviada, eles podem continuar a processar. Isso permite a entrega de mensagens quando o receptor ou emissor não estão activos ou disponíveis na altura da execução. Uma aplicação que envia mensagens não tem a garantia que a sua mensagem vai ser lida, nem  a garantia de quanto tempo vai demorar até que seja entregue. 
subsection{Padrões de troca de mensagens }
\begin{itemize}
\item \textbf{Ponto a Ponto}\\
O modelo de mensagens ponto a ponto fornece uma troca assíncrona de mensagens entre aplicações. Neste modelo, as mensagens de um cliente produtor são encaminhadas para um cliente consumidor através de uma \textit{queue}. O mecanismo mais comum de \textit{queue} é uma \textit{queue} FIFO, na qual as mensagens são ordenadas conforme a ordem em que são recebidas pelo sistema de mensagens, assim que são consumidas são removidas do topo da \textit{queue}. 
Apesar de não existir uma restrição para o número de clientes que podem publicar numa \textit{queue}, existe normalmente apenas um cliente consumidor, apesar de não ser um requisito muito rígido. Cada mensagem que é entregue apenas uma vez a apenas um receptor. O modelo permite que múltiplos receptores possam se conectar à queue, mas apenas um dos receptores vai consumir a mensagem. No modelo ponto a ponto, as mensagens são sempre entregues e são guardadas na \textit{queue} até que um consumidor estar preparado para a consumir.  
\item \textbf{Publish/Subscribe}
O mecanismo  de \textit{Publish/Subscribe} é um mecanismo muito poderoso, usado para desiminar informação entre produtores e consumidores anonímos de mensagens. Podem ser mecanismos de distribuição de um para um ou de muitos para muitos, permitem a um simples consumidor enviar e receber mensagens de, potencialmente, centenas de milhares de utilizadores.\\
No modelo \textit{publish/subscribe}, a aplicação emissora e receptora é livre da necessidade de perceber alguma coisa sobre a aplicação alvo.  Só tem necessidade de enviar a informação para um destino dentro da máquina \textit{publish/subscribe}. A máquina vai enviar prosteriormente para o consumidor. Os cliente produzem mensagens para um tópico especifico ou para um canal que são subscritos por cliente que pretendem consumir essas mensagens. O serviço mapeia as mensagens para os consumidores consoante os tópicos que eles estão interessados. Dentro do modelo \textit{publish/subscribe}, não existe a restrição no papel de um cliente, ele pode ser tanto consumidor como produtor de um tópico/canal.\\
\end{itemize}


% ---------------------- Websockets

\subsection{Introdução a websockets}

A conceção inicial da web considerou apenas a comunicação cliente-servidor num sentido apenas. Actualmente o HTML5 procura corrigir esta entrave, contudo ainda muitos projectos utilizam \textit{long-polling} para simular a comunicação cliente-servidor.

Actualmente os browsers são actualizados regularmente e suportam a API de comunicação do HTML5.
\subsubsection{\textit{Long pulling}}

\begin{figure}[H]
\centering
\includegraphics[width=0.9\textwidth]{longpolling-architecture.png}
\caption{Esquema de \textit{long pulling}}
\label{fig:long_pulling}
\end{figure}

Um cliente (browser) envia por HTTP um pedido para o servidor com o identificador do utilizador (por exemplo) e do estado actual. No servidor é criado um processo que repetidamente verifica na base de dados se existe um estado novo. Quando existe um novo estado o cliente recebe e envia um novo pedido ao servidor.

\subsubsection{\textit{Server-Sent Events}}

\begin{figure}[H]
\centering
\includegraphics[width=0.9\textwidth]{sse-architecture.png}
\caption{Esquema de \textit{server-sent events}}
\label{fig:sse-architecture}
\end{figure}

Um cliente (browser) faz um pedido ao servidor. O servidor responde com o último estado na base de dados. O cliente recebe a resposta e em três segundos (por exemplo) envia um novo pedido.

\subsubsection{Websockets}

\begin{figure}[H]
\centering
\includegraphics[width=0.9\textwidth]{websocket-architecture.png}
\caption{Esquema de \textit{websockets}}
\label{fig:websockets-architecture}
\end{figure}

Um cliente notifica o servidor de websockets de um evento. O servidor imediamente notifica todos os clientes ativos do evento. Este processo pode envolver filtros e subscrição de eventos.

\input{websockets}

\subsection{Ruby}
\label{sec:ruby}

Ruby é uma linguagem de programação intrepertada com suporte para diferentes paradigmas (funcional, orientado a objectos e imperativo). Desde o lançamento em 1995 Ruby tem crescido em comunidade e potencial. O seu criador, Yukihiro Matsumoto, pretendia uma linguagem que qualquer programador pudesse aperciar:

\begingroup
\leftskip4em
\rightskip\leftskip
``Ruby is simple in appearance, but is very complex inside, just like our human body.'' \cite{matz}
\par
\endgroup

Ruby encontra-se actualmente na versão 2.0.0-p195 no entanto o projecto foi desenvolvido na versão \textbf{2.0.0p0}.

\subsubsection{\textit{Global Interpreter Lock}}
\label{sec:gil}

Existem diversos intrepertadores para Ruby sendo os mais importantes \textbf{Matz's Ruby Interpreter (MRI)}, em homenagem ao criador, e \textbf{JRuby}, implementado no topo da Java Virtual Machine.
A diferença mais relevante entre ambos para este projecto é o \textit{Global Interpreter Lock} (GIL) que existe no MRI. O GIL é uma camada responsável por proteger o intrepertador contra código \textit{non thread-safe}.

A figura~\ref{fig:ruby-gil} apresenta uma comparação entre três versões do Ruby. Na versão 1.8 o intrepertador Ruby possuí apenas uma thread do sistema para execussão. Já na versão 1.9, e também na versão 2 ainda que não seja visível na imagem, várias threads do sistema são alocadas ao intrepertador, o que parece prometer paralelismo de execussão. No entanto em ambos os casos existe a camada do GIL que protegendo contra a execussão de código \textit{non thread-safe} permite que apenas uma thread seja executada de cada vez pelo processador, ou seja, ambas as versões do Ruby correm num core do CPU apenas.
Por outro lado a implementação em JRuby não contém a camada GIL o que abre as portas ao paralelismo das aplicações em Ruby.

\begin{figure}[H]
\centering
\includegraphics[width=0.9\textwidth]{xruby_gil.png}
\caption{\textit{Global Interpreter Lock}}
\label{fig:ruby-gil}
\end{figure}

Parecendo um cenário desvantajoso para o MRI é de notar que existem soluções para contornar este problema. Se se pensar em processos em vez de threads por procurar-se outros meios de repartir trabalho. Decompondo a aplicação e adicionando meios de comunicação entre processos (Starling, RabbitMQ, outros) consegue-se que multiplos processos da mesma aplicação executem concurrentemente.
