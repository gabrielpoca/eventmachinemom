\section{Arquitectura}
O servidor é composto por brokers e uma base de dados. 
Os brokers são instâncias da mesma aplicação idênticos, excepto nos endereços onde atendem ligações. A ideia é que seja indiferente o broker ao qual um cliente estabelece ligação uma vez que o comportamento é distribuído pelos restantes.
Por exemplo: dois clientes subscreveram o mesmo canal em brokers diferentes e um terceiro cliente envia uma mensagem para esse canal, então ambos os novos clientes recebem a mensagem.

\begin{figure}[H]
\centering
\includegraphics[width=0.65\textwidth]{brokers.png}
\caption{\textit{Visão simplista da arquitectura do servidor.}}
\label{fig:brokers-arq}
\end{figure}

A figura~\ref{fig:brokers-arq} apresenta um esquema simplista das ligações entre os diferentes componentes do servidor. Os brokers estão todos ligados entre si e cada um está por sua vez ligado à base de dados.

A base de dados é responsável por armazenar as mensagens persistentes, manter uma sequência que forneça identificadores únicos para identificar os clientes e registar os endereços dos servidores activos. Todas as restantes acções são da responsabilidade dos brokers, deste modo a ligação entre todos os brokers serve dois propósitos:

\begin{enumerate}
\item \textbf{Difundir mensagens}. As mensagens que um broker recebe são difundidas pelos restantes. Isto permite que vários clientes possam subscrever os mesmos canais quando estão ligados em brokers diferentes.
\item \textbf{Informar da actualização na lista de servidores}. Quando existe um novo broker online ou um dos brokers é identificado como inactivo o novo broker ou o broker que identificou o problema, respectivamente, actualizam a base de dados e enviam uma mensagem de actualização a todos os brokers.
\end{enumerate}

