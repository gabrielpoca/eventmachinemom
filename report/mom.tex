\subsection{Message Oriented Middleware - MoM}

\label{sec:mom}
Entende-se por \textit{message oriented middleware} (MOM) como um método de comunicação entre componentes de sistemas distribuídos.\\

Um \textit{Message Oriented Middleware}, ou simplesmente MoM é uma arquitectura que fornece uma camada entre as aplicaçães, substituindo a comunicação directa entre as mesmas por um sistema de troca de mensagens.\\
Uma implementação \textit{MoM} oferece uma API que fornece um nível de abstracção capaz de aumentar a portabilidade, interoperabilidade e flexibilidades das aplicações que correm sobre a MOM.Os programadores são assim libertados dos detalhes das várias plataformas e protocolos, reduzindo a complexidade das comunicações das suas aplicações.\\
A comunicação é efectuada através de troca de mensagens pela estrutura típica cliente/servidor(usando \textit{broadcast} ou \textit{multicast}) ou entre pilhas mantidas por gestores locais. Os gestores de pilhas são poderosos em termos de aplicabilidade e versatilidade.\\ 
Os sistemas que usam MOM providenciam comunicação distribuída com base num modelo de interacção assíncrono. Os participantes do sistema não precisam de bloquear e esperar numa mensagem enviada, eles podem continuar a processar. Isso permite a entrega de mensagens quando o receptor ou emissor não estão activos ou disponíveis na altura da execução. Uma aplicação que envia mensagens não tem a garantia que a sua mensagem vai ser lida, nem  a garantia de quanto tempo vai demorar até que seja entregue. 
subsection{Padrões de troca de mensagens }
\begin{itemize}
\item \textbf{Ponto a Ponto}\\
O modelo de mensagens ponto a ponto fornece uma troca assíncrona de mensagens entre aplicações. Neste modelo, as mensagens de um cliente produtor são encaminhadas para um cliente consumidor através de uma \textit{queue}. O mecanismo mais comum de \textit{queue} é uma \textit{queue} FIFO, na qual as mensagens são ordenadas conforme a ordem em que são recebidas pelo sistema de mensagens, assim que são consumidas são removidas do topo da \textit{queue}. 
Apesar de não existir uma restrição para o número de clientes que podem publicar numa \textit{queue}, existe normalmente apenas um cliente consumidor, apesar de não ser um requisito muito rígido. Cada mensagem que é entregue apenas uma vez a apenas um receptor. O modelo permite que múltiplos receptores possam se conectar à queue, mas apenas um dos receptores vai consumir a mensagem. No modelo ponto a ponto, as mensagens são sempre entregues e são guardadas na \textit{queue} até que um consumidor estar preparado para a consumir.  
\item \textbf{Publish/Subscribe}
O mecanismo  de \textit{Publish/Subscribe} é um mecanismo muito poderoso, usado para desiminar informação entre produtores e consumidores anonímos de mensagens. Podem ser mecanismos de distribuição de um para um ou de muitos para muitos, permitem a um simples consumidor enviar e receber mensagens de, potencialmente, centenas de milhares de utilizadores.\\
No modelo \textit{publish/subscribe}, a aplicação emissora e receptora é livre da necessidade de perceber alguma coisa sobre a aplicação alvo.  Só tem necessidade de enviar a informação para um destino dentro da máquina \textit{publish/subscribe}. A máquina vai enviar prosteriormente para o consumidor. Os cliente produzem mensagens para um tópico especifico ou para um canal que são subscritos por cliente que pretendem consumir essas mensagens. O serviço mapeia as mensagens para os consumidores consoante os tópicos que eles estão interessados. Dentro do modelo \textit{publish/subscribe}, não existe a restrição no papel de um cliente, ele pode ser tanto consumidor como produtor de um tópico/canal.\\
\end{itemize}
