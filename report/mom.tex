\section{Message Oriented Middleware - MoM}
Um \textit{Message Oriented Middleware}, ou simplesmente MoM é uma arquitectura que fornece uma camada entre as aplicaçães, substituindo a comunicação directa entre as mesmas por um sistema de troca de mensagens.\\
Uma implementação \textit{MoM} oferece uma API capaz de funcionar com um número relativamente vasto de plataformas e redes. Essa API fornece um nível de abstracção capaz de aumentar a portbailidade, interoperabilidade e flexibilidades das aplicações que correm sobre a MOM.\\
Usando a API, os programadores são libertados dos detalhes das várias plataformas e protocolos, reduzinhdo assim a complexidade da implementação das comunicações das suas aplicações.\\
A comunicação é efectuada através de mensagens que são transmitidas, ou pela estrutura típica cliente/servidor(usando \textit{broadcast} ou \textit{multicast}), ou entre pilhas mantidas por gestores locais. A alternativa que usa gestores de pilhas é a mais poderosa em termos de aplicabilidade e versatilidade.\\ 
Os sistemas que usam MOM providencias comunicação distribuída com base num modelo d einteracção assíncrono. Os participantes do sistema não precisam de bloquear e espearar numa mensagem enviada, eles podem continuar a processar assim que uma mensagem for enviada. Isso permite a entrega de mensagens quando o receptor ou emissor não estejam activos ou disponíveis para responder na altura da execução. Uma aplicação que envia mensagens não tem a garantia que a sua mensagem vai ser lida por outra aplicação, nem sequer tem a garantia de quanto tempo vai demorar até que a sua mensagem seja entregue. 
\subsection{Modelos de Mensagens}
\subsubsection{Ponto a Ponto}
O modelo de mensagens ponto a ponto fornece uma troca assíncrona de mensagens entre aplicações. Neste modelo, as mensagens de um cliente produtor são encaminhadas para um cliente consumidor através de uma \textit{queue}. O mecanismo mais comun de \textit{queue} é uma \textit{queue} FIFO, na qual as mensagens são ordenadas conforme a ordem em que são recebidas pelo sistema de mensagens, assim que são consumidas são removidas do topo da \textit{queue}. 
Enquanto não existe uma restrição para o número de clientes que podem publicar numa \textit{queue}, existe normalmente apenas um cliente consumidor, apesar de não ser um requisito muito rígido. Cada mensagem que é entregue apenas uma vez a apenas um receptor. o modelo permite que múltiplos receptores possam se concetar à queue, mas epnas um dos receptores vai consumir a mensagem. As técnicas de usar múltiplos clientes para ler de uma \textit{queue} pode  
\subsubsection{Publish/Subscribe}
O mecanismo  de \textit{Publish/Subscribe} é um mecanismo muito poderoso, usado para desiminar informação entre produtores e consumidores anonímos de mensagens. Podem ser relaçoes de um para um ou de muitos para muitos, permitem a uma simples consumidas enviar e receber mensagens de potencilamente centenas de milhares de utilizadores. \\
No modelo publish


