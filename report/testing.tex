\section{Testes}

No contexto deste problema o que se pretende é apresentar resultados quanto ao desempenho da aplicação. Como tal os testes foram realizados de forma a comparar a latência de receção das mensagens e débito de mensagens da aplicação por numero de clientes.

Para realizar os testes utilizaram-se duas máquinas. A primeira máquina, com um Intel Core i5 (dois cores) e velocidade 2.4GHz, executa três brokers. A segunda máquina, com um core de 800MHz, ficou responsável por executar os clientes.

Facilmente se concluiu que estas não estão reunidas as melhores condições para validar o desempenho da aplicação:
\begin{enumerate}
\item O propósito dos brokers distribuídos pode ser justificado pelo facto de tirar partido de múltiplos cores no entanto o objetivo é permitir a execução em máquinas diferentes.
\item Cada cliente de teste executa uma aplicação ruby, ou seja, dez clientes são dez processos numa máquina de core único, e assim sucessivamente. Ou seja tanto o desempenho dos clientes não é o adequado como o próprio tráfego todo para a mesma máquina condiciona os resultados.
\end{enumerate}

Um dos brokers é responsável por contabilizar o débito de mensagens. Quanto aos clientes apenas um envia mensagens e todos os restantes recebem. Dos que recebem um é responsável pelo cálculo da latência.

Note-se que por cada mensagem submetida num broker é difundida pelos restantes, como tal mil mensagens são o equivalente a 2000 mensagens trocadas entre os brokers.
Desta experiência as conclusões que podem tirar-se não são surpreendenes, o comportamento apresentado é o esperado e em termos de comparação de desemepenho seria necessário realizar os mesmos testes noutras aplicações.
