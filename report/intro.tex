\section{Introdução}

Este documento serve o propósito de documentar o estudo e desenvolvimento realizado no projeto integrado da unidade curricular de sistemas distribuídos.

O que se pretende com este projeto é desenvolver um \textit{message-oriented middleware} (MOM) distribuído que suporte comunicação por websockets. 
A motivação para o projeto está tanto na crescente utilização de clientes web como na ausência de suporte nativo para websockets por parte dos MOM mais populares. Soluções como o RabbitMQ e ZeroMQ fornecem plugins que criam uma ponte entre websockets e o protocolo Simple Text Oriented Message Protocol.

Uma vez que se pretende explorar uma área extensa algumas funcionalidades presentes em grande parte dos MOM têm de ser deixadas de parte, focando a implementação nas caracteristicas que suportam a análise que se pretende neste projeto:

\begin{itemize}
\item Suportar múltiplos clientes.
\item Permitir que os clientes subscrevam canais.
\item Permitir criar canais persistentes.
\end{itemize}

No contexto deste projeto o termo \textbf{canal} canal significa \textbf{message queue}, este assunto será explorado com mais detalhe na secção~\ref{sec:mom}.


As próximas secções desta introdução introduzem conceitos e tecnologias utilizadas com o objetivo de suportar as análises posteriores no relatório.


