\subsection{Websockets}
WebSockets é um protocolo que permite umq comunicação \textit{full-duplex} sobre um único \textit{socket} TCP. Foi desenhado para ser usado em \textit{browsers} e em servidores que suportem o HTML 5.\\
Historicamente, ao criar uma aplicação \textit{web} em que fosse necessário uma comunicação bidireccional entre cliente e servidor, isso implicava um abuso do HTTP para o \textit{poll} do servidor para actualizações enquanto enviava notificações para chamadas HTTP distintas. Esta abordagem levantava vários problemas, entre os quais o facto de o servidor ser forçado a usar um número diferente de conexões TCP subjacentes para cada cliente assim como facto de o protocolo ter um grande \textit{overhead}.\\
A solução mais simples para o problema seria usar apenas uma conxeão TCP para o tráfego em ambas as direcções. Essa solução é apresentada pela API fornecida pelos \textit{WebSockets} ao providenciar uma alternativa ao usual HTTP \textit{polling} para comunicação nas duas direcções.\\
Esta técnica de \textit{WebSockets} pode ser usada para várias aplicações Web, especialmente as que necessitam de constante trocas de pacotes em tempo real como jogos on-line. O protocolo websocket é um tecnologia incluída no html5 e é suportada actualmente pela maior parte dos browsers. O seu esquema de de url é \textit{ws} ou \textit{wss} para comunicações seguras. \\

A conceção inicial da web considerou apenas a comunicação cliente-servidor num sentido apenas. Actualmente o HTML5 procura corrigir esta entrave, contudo ainda muitos projectos utilizam \textit{long-polling} para simular a comunicação cliente-servidor.\\

Actualmente os browsers são actualizados regularmente e suportam a API de comunicação do HTML5.
\subsubsection{\textit{Long pulling}}

\begin{figure}[H]
\centering
\includegraphics[width=0.9\textwidth]{longpolling-architecture.png}
\caption{Esquema de \textit{long pulling}}
\label{fig:long_pulling}
\end{figure}

Um cliente (browser) envia por HTTP um pedido para o servidor com o identificador do utilizador (por exemplo) e do estado actual. No servidor é criado um processo que repetidamente verifica na base de dados se existe um estado novo. Quando existe um novo estado o cliente recebe e envia um novo pedido ao servidor.

\subsubsection{\textit{Server-Sent Events}}

\begin{figure}[H]
\centering
\includegraphics[width=0.9\textwidth]{sse-architecture.png}
\caption{Esquema de \textit{server-sent events}}
\label{fig:sse-architecture}
\end{figure}

Um cliente (browser) faz um pedido ao servidor. O servidor responde com o último estado na base de dados. O cliente recebe a resposta e em três segundos (por exemplo) envia um novo pedido.

\subsubsection{Websockets}

\begin{figure}[H]
\centering
\includegraphics[width=0.9\textwidth]{websocket-architecture.png}
\caption{Esquema de \textit{websockets}}
\label{fig:websockets-architecture}
\end{figure}

Um cliente notifica o servidor de websockets de um evento. O servidor imediamente notifica todos os clientes ativos do evento. Este processo pode envolver filtros e subscrição de eventos.






