\section{WebSockets}
WebSockets é um protocolo que permite umq comunicação \textit{full-duplex} sobre um único \textit{socket} TCP. Foi desenhado para ser usado em \textit{browsers} e em servidores que suportem o HTML 5.\\
Historicamente, ao criar uma aplicação \textit{web} em que fosse necessário uma comunicação bidireccional entre cliente e servidor, isso implicava um abuso do HTTP para o \textit{poll} do servidor para actualizações enquanto enviava notificações para chamadas HTTP distintas. Esta abordagem levantava vários problemas, entre os quais o facto de o servidor ser forçado a usar um número diferente de conexões TCP subjacentes para cada cliente assim como facto de o protocolo ter um grande \textit{overhead}.\\
A solução mais simples para o problema seria usar apenas uma conxeão TCP para o tráfego em ambas as direcções. Essa solução é apresentada pela API fornecida pelos \textit{WebSockets} ao providenciar uma alternativa ao usual HTTP \textit{polling} para comunicação nas duas direcções.\\
Esta técnica de \textit{WebSockets} pode ser usada para várias aplicações Web, especialmente as que necessitam de constante trocas de pacotes em tempo real como jogos on-line. O protocolo websocket é um tecnologia incluída no html5 e é suportada actualmente pela maior parte dos browsers. O seu esquema de de url é \textit{ws} ou \textit{wss} para comunicações seguras. \\
O protocolo \textit{WebSocket} foi desenhado para suplantar as tecnologias de comunicação bidireccional existenetes que usam o HTTP como camada  de transporte. Essas tecnologias são implementadas como \textit{trade-offs} entre eficiência e confiança porque o HTTP não foi inicialmente desenhado para ser usado em conexões bidireccionais. O protocolo \textit{WebSocket} procura atingir os objectivos das tecnologias bidireccionais do HTTP no contexto do próprio HTTP, é desenhado para trablhar sobre as portas 80 e 443 do HHTP assim como suportar HTTP \textit{proxies} e intermediários, mesmo que isso implique uma maior complexidade.\\
\subsection{API}
Construtor WebSocket
\begin{verbatim}
var connection = new WebSocket('ws://example.org');
\end{verbatim}
Se a conexão for aceite, então ocorre o evento \textit{onopen} do lado do cliente
\begin{verbatim}
connection.onpen=function(){
	console.log('Connection open');
}
\end{verbatim}
Se a conexão é recusada do lado do servidor, o método \textit{onclose} ocorre
\begin{verbatim}
connection.onclose = function(){
	console.log('Connection closed');
}
\end{verbatim}
Fechar explicitamente o socket
\begin{verbatim}
connection.close()
\end{verbatim}
Enviar uma mensagem
\begin{verbatim}
connection.send('Hello World')
\end{verbatim}
Evento receber mensagem
\begin{verbatim}
connection.onmessage = function(e){
	var server_message = e.data;
	console.log(server_message);
}
\end{verbatim}

